\section{Discussion}
\subsection{Part 1}
In this experiment, there are two main part to be studied. First, basics of standard model processes and for this purpose mystery dataset events were investigated. As a result, $\slashed{E}_T$ and neutrino relations, pair production of leptons indicated by the Drell-Yan process, hard scattering processes, hadronization and jets were analyzed. How different elemantary particles in standard model reacts with ATLAS detector is also investigated and searched for distinctions. These distinctions were made based on the particle tracks on the detector and kinematic parameters. As a result of the analysis for events in the mystery dataset, some processes were excluded and the remaining were physically justified.
Second part based on the measuring the $W$ boson mass and for this energy measurements on ECAL must be calibrated. It was observed that raw energy measurements did not give reliable results and detector calibration was a necessity. The different response of the ECAL modules used in electron energy measurement and the systematic low measurements caused by some energy losses confirmed the need for calibration. $Z \rightarrow e^+e^-$  decays were used for this purpose. The obtained invariant mass distribution was calibrated to be compatible with the $Z$ boson mass in the literature. This accuracy directly contributed to the mass measurement of the W boson.

\subsection{Part 2}
In the second part of the experiment, the mass of the $W$ boson was determined. In order to do so, different kinematic values were analyzed, to get an understanding of 
the distributions of data which are actually measured in a detector. From this, the QCD scale and its effect on the distributions of simulated data was analyzed,
since the integrated luminosity of the background is unknown. After obtaining the optimal QCD scale factor to get the best agreement between simulated and real ATLAS data,
a cut selection on the kinematic variables was done to reduce the background as much as possible. With the cuts and QCD scale factor applied,
the Jacobi-peaks of the transverse electron momentum distributions for different data sets could be extracted using a fit.
With these half maximum points and the corresponding $W$ mass, a gauge curve was constructed to obtain the $W$ boson mass of the real ATLAS data set.
Different Uncertainties were evaluated and the final result is a $W$ boson mass of 
\begin{align*}
    m_W = (80.391 \pm 0.215(\mathrm{stat}) \pm 0.526 (\mathrm{sys}))\,\mathrm{GeV}.
\end{align*}
This result is very close to the literature value. The uncertainty, especially the systematic uncertainty, is relatively large,
due to reasons explained in section \ref{sec:uncertainties}. Overall the result of the analysis is solid. 